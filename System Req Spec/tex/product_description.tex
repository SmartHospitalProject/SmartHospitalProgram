The Smart Hosptial Management Tools website is a culmination of multiple websites and software tools used by UT Arlington's Smart Hospital. The product focuses on procedure and work scheduling, allowing users of the tools to track, who, what, and where aspects are scheduled for a hospital procedure, along with inventory management.

\subsection{Features \& Functions}
The homepage of the toolset will allow a user to visit sections of the website that they are allowed access to. A student, for instance, should be able to visit the scheduling area and request procedures that would be approved by a manager or director. A management section, as seen in the early first concept screen shot, permits item tracking and medicine stock views. A calendar will contain scheduled events that keep track people and items that may be used at any specific time. Faculty workers can also use the website to clock in and clock out at manager specified locations based on ip addresses. Finally, a portion of tools will allow managers and teachers to look over reports including worker schedules, hours worked, time for procedures, and procedure participants.

\subsection{External Inputs \& Outputs}
Describe critical external data flows. What does your product require/expect to receive from end users or external systems (inputs), and what is expected to be created by your product for consumption by end users or external systems (outputs)? In other words, specify here all data/information to flow into and out of your systems. A table works best here, with rows for each critical data element, and columns for name, description and use.
Users shall be able to create accounts.

\subsection{Product Interfaces}
The main input from users will be user account creation, 
Specify what all operational (visible) interfaces look like to your end-user, administrator, maintainer, etc. Show sample/mocked-up screen shots, graphics of buttons, panels, etc. Refer to the critical external inputs and outputs described in the paragraph above.